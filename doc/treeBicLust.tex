\documentclass{article}
%
%\usepackage{pdfsync}           % Used in Mac OSX
\usepackage[mathcal,mathbf]{euler}
\usepackage{theorem,amsmath,enumerate,fancyhdr,amssymb,amsfonts}
\usepackage{graphicx}           %Used in Mac OSX
%\usepackage[pdftex]{graphics}  %Might be needed in non Mac OSX systems
\usepackage{myDefs}
\usepackage{url}
%In order to be consistent use the following notation in your notes:
\graphicspath{{./fig/}}
\title{
Tissue tree gene biclustering and gene dynamics
}
\author{Avinash Das}

\date{\today}
\begin{document}

\pagestyle{fancy}
\lhead{{\bf Tissue tree biclustering}{ }
\\{\bf } } 
\rhead{{\bf Date: }\today}

\maketitle

\section{Problem}
Given the gene expression across multiple tissues, find gene clusters and most probable tree defining relations between tissues.  
Tissue tree implies the relation of the different tissues in an embryo, with embryo as root. Cluster of gene implies  
group of genes homogeneously all active or inactive in tissues of subtree. For eg. one of intended outcome would be genes those
are homogeneously active genes in tissue subtree with heart as the root, i.e. these genes will be active in left ventricle, right ventricle
and all tissues related to heart.


\section{Data}
The data is collected from http://www.eurexpress.org/, composed of expression of 5600 genes across 811 tissues. The genes are manually annotated and have discreet value as strong, medium, weak or no expression. 
For initial analysis we can treat the data as binary i.e. expressed or not expressed. Later it can be generalized to take four discreet values. 

\section{Generative process}

%\subsection{Defining latent variable Z}
\par \textbf{Latent state variable $Z$}:
Given a tree $\pi(V,E)$ and genes $G$, we define latent state variable $Z= \{z_{ju} \in\{0,1,2\} : \forall j \in G, u \in V\}$ as:
%corresponding to each gene at each node:
\begin{eqnarray}
	z_{ju} &=& 0 \implies \textit{gene $j$ is homogenously inactive in subtree with $u$ as root.} \nonumber\\ 
	z_{ju} &=& 1 \implies \textit{gene $j$ is homogenously active with high expression in subtree with  $u$ as root.} \nonumber \\
	z_{ju} &=& 2 \implies \textit{gene $j$ have inhomogenously activity in subtree with $u$ as root.} \nonumber 
\end{eqnarray}
In addition, once a node $z_{ju}$ takes a value either 0 or 1 corresponding to states homogeneously active or in-active all its children 
node in subtree stays in same homogeneous state. 

Prior over tissue tree $\pi$ can be defined by Coalescent clustering. 

$Z$ can be then sampled from multinomial distribution with a Dirichlet prior.
\begin{eqnarray}
	\pi &\sim& \text{coalescent}()\\
	Z|\pi,\phi &\sim& \text{mult}(\phi) \nonumber \\
	\phi &\sim&  \text{Dir}(\alpha,\beta,\gamma) \nonumber
	\label{eqn:zdist1}
\end{eqnarray}

\par \textbf{Generation of expression level}: 
Given the state variable $Z$, gene expression is generated from a Bernoulli with parameter only dependent on state variable $Z$. 
\begin{eqnarray}
	Y|Z = i &\sim& \text{Bin}(p_i) \nonumber \\
	p_i &\sim&  \text{Beta}(\alpha_i,  \beta_i) \forall i \in \{1,2,3\}  
	\label{eqn:ydist}
\end{eqnarray}
The beta priors can be chosen so to reflect our expected behaviour of expression level in different gene state. One example of prior that can be taken
is shown in fig. \ref{fig:betaP}.
%\begin{figure}[ht]
	%\begin{center}
		%\includegraphics[scale=0.5]{beta.jpg}
	%\end{center}
	%\caption{Beta priors for homogeneous active, homogeneous inactive and inhomogeneous gene state. The priors are consistent
	%with the interaction matrix data; e.g. $E(p_2)$ = average that any gene is expressed in interaction matrix.}
	%\label{fig:betaP}
%\end{figure}


%The method can be extendedfor the discrete values of expression level by using multinomial distribution instead of binomial distribution in order to represent the different expression stages (weak, intermediate, high and no expression). 
\section{Expected outcomes}
\begin{itemize}
	\item We will obtain a tissue tree. 
	\item At each node of the tree we will get cluster of genes which are active or inactive in the subtree.
	\item $\delta$ parameter in Kingman's coalescent will give the distance between the each tissue. For eg. we will get information if brain is closer to heart
		than kidney. 
	\item Probably clustering model will be resistant to error due manual annotation.
\end{itemize}


%% for presentation
\begin{eqnarray}
	\pi &\sim& \text{kingsman}() \nonumber\\
	Z|\pi,\phi &\sim& \text{mult}(\phi) \nonumber \\
	\phi &\sim&  \text{Dir}(\alpha,\beta,\gamma) \nonumber\\
	Y|Z = i &\sim& \text{Bin}(p_i) \nonumber \\
	p_i &\sim&  \text{Beta}(\alpha_i,  \beta_i) \nonumber
	\label{eqn:pptzdist1}
\end{eqnarray}
\begin{eqnarray}
	q(\rho_{l_i r_i}, \delta_i) \propto exp\left( -\binom{n-i +1}{2}\delta_i \right) \tilde{Z}_{i}(X|\theta_i) \nonumber \\
	\tilde{Z}_{i}(X|\theta_i) = \int \int p(a) k_{- \infty t_{i}} (x,y) M_{i}(y) dy da \nonumber\\ 
	M_i(y) \propto \prod_{b=l,r} \int k_{t_i t_{bi}}(y,y_b) M_i(y_b) dy_b \nonumber
\end{eqnarray}

\begin{equation}
	Y|Z,\pi &\sim& \text{Beta}(\alpha_i, \beta_i) \nonumber
\end{equation}

\begin{eqnarray}
	\boldsymbol{\alpha_{i}} &=& \boldsymbol{\delta P}(x_i)  \quad \forall i \in \texit{leaf} \\ 
	\boldsymbol{\alpha_i} &=& \boldsymbol{\alpha_l \alpha_r \Gamma P}(x_i)  \quad\forall i \not{\in} \texit{leaf} 
\end{eqnarray}

\begin{eqnarray*}
	\textit{E step} \quad
	\hat{v}_{jk;i}(t) &=&  Pr(C_{l} = j, C_{r} = k,  C_{t} = i | \boldsymbol{y^{T}}) \\
	  &=&  \alpha_l(j) \alpha_r(k) \gamma_{jk;i} p_k(x_t) \beta_t(k)/L_t\\
	  \textit{M step:} \quad \gamma_{jk;i} &=&\tilde{f}_{jk;i} \\
	  f_{jk;i} &=& \sum \hat{v}_{jk;i}(t)
\end{eqnarray*}


\section{Introduction}
\subsection{Classification of genes}
\subsection{Tissue tree}
\subsection{In-situ hybridization data}

\section{Methods}

\subsection{Kingman's coalescent}
To infer tree from observed data, we used agglomerative clustering using Kingman's coalescent \cite{teh2008bayesian}.
It defines exponential distribution as prior over trees.  It calculates posterior of the trees given the observation.
The inference is performed by message passing algorithm, by passing message upward from leaf to trees. One of the advantage of
method is the proposed distribution is exchangeable.

Let $\pi$ be the tree defining genaology of $n$ individuals. The tree $\pi$ can be defined by $n-1$ coalescent events. 
The $i$th coalescing event is defined by $\rho_{l_i}$ and $\rho_{r_i}$, the left and right subtree that are coalescing  
at waiting $\delta_i$ after $i-1$th coalescing event. The prior is defined by:
\begin{equation}
	p(\pi) = \prod_{i}^{n-1}exp\left( -\binom{n-i +1}{2}\delta_i \right)
	\label{ref:prior}
\end{equation}


\cite{teh2008bayesian} proved that joint probability of observation and tree can be given by:
\begin{eqnarray}
	p(x, \pi) &\propto& \prod_{i}^{n-1}exp\left( -\binom{n-i +1}{2}\delta_i \right) \tilde{Z}_{i}(X|\theta_i) \\ 
	\textit{where, } \tilde{Z}_{i}(X|\theta_i) & =& \int \int p(a) k_{- \infty t_{i}} (x,y) M_{i}(y) dy da \\ 
	M_i(y) &\propto& \prod_{b=l,r} \int k_{t_i t_{bi}}(y,y_b) M_i(y_b) dy_b 
	\label{eqn:proposal}
\end{eqnarray}
$M_i(y)$ is message propagated upward to subtree $\theta_i$ from its both children. This can be calculated iteratively
by propagating messages from leaf to root node. $\tilde{Z}_{i}(X|\theta_i)$ can be viewed as local likelihood. Inference
is based on \ref{eqn:proposal}. At each step $i$, a duration is $\delta_i$ is sampled and then a pair $\rho_{l_i},\rho_{r_i}$
is chosen.


\subsection{Latent gene state variable $Z$}
Given we have reasonable accurate tissue tree representing the relationship between tissues,
the problem of gene biclustering reduces to inferring the 
state of genes at each internal node of the inferred tree. 
Given a tree $\pi(V,E)$ and genes $G$, we define latent state variable $Z= \{z_{ju} \in\{0,1,2\} : \forall j \in G, u \in V\}$ as:
%corresponding to each gene at each node:
\begin{eqnarray}
	z_{ju} &=& 0 \implies \textit{gene $j$ is homogenously inactive in subtree with $u$ as root.} \nonumber\\ 
	z_{ju} &=& 1 \implies \textit{gene $j$ is homogenously active with high expression in subtree with  $u$ as root.} \nonumber \\
	z_{ju} &=& 2 \implies \textit{gene $j$ have hetrogenous activity in subtree with $u$ as root.} \nonumber 
\end{eqnarray}
Once a node $z_{ju}$ takes a value either 0 or 1 corresponding to states  
homogeneously active or in-active all its children node in subtree stays in same homogeneous state.

\subsection{Generative process I}
The problem of biclustering of the genes 
reduces to simultaneously inferring 1) tree of tissue and 2) gene state variable at each internal node of the tree.
We define a generative process to generate gene expression data given the tree using $Z$ as the internal latent variable:
\begin{eqnarray}
	\pi &\sim& \text{kingsman}() \nonumber\\
	Z|\pi,\phi &\sim& \text{mult}(\phi) \nonumber \\
	\phi &\sim&  \text{Dir}(\alpha,\beta,\gamma) \nonumber\\
	Y|Z = i &\sim& \text{Bin}(p_i) \nonumber \\
	p_i &\sim&  \text{Beta}(\alpha_i,  \beta_i)
	\label{eqn:gen1}
\end{eqnarray}
The basic idea of generative model is very similar to Kingman's agglomerative clustering. The main difference is instead of using 
observed data (gene expression) to infer tree, the generative process uses the state variable $Z$ at each internal nodes. 
Therefore, gene expression is thought to be generated at leafs given the tree and gene state variable.


The message passing algorithm in \cite{teh2008bayesian} uses transition kernel $ k_{t_i t_{bi}}(y,y_b)$
to define transition from one state to other. If 
$Z$ is used to infer tree, transitions cannot be independent for both children. That is transition of left children to
parent cannot be independent of state of right children. For instance:  once a node become homogeneous all it descendant 
should be homogeneous. This kind of complicated dependencies demand a three dimensional transition kernel. Such kernel does not
arrest a closed form solution of messages from equation \ref{eqn:proposal}.

\subsection{Generative process II}
To solve the problem described in previous section, we propose to split the tree inference with the inference of latent 
gene state $Z$. We infer tree directly from the observed gene expression from agglomerative clustering \cite{teh2008bayesian}.
At each of the internal state we also infer gene expression $Y$ at each internal nodes of tree by passing message downward and 
combing it with upward message. 

Given the tree $\pi$ and inferred gene expression at each internal node. We define a new generative process:

\begin{equation}
	Y|Z = i, \pi &\sim& \text{Beta}(\alpha_i,  \beta_i)
	\label{eqn:gen2}
\end{equation}

The $Y$ at internal nodes are probability therefore it arrest a $Beta$ distribution. We can also assume given
the tree structure and expression at internal nodes, gene expression becomes independent. This implies
we can infer $Z$ independently for each genes. 

\subsection{Inference}
We used message passing algorithm to compute posterior probabilities of latent variable at internal nodes.
Parameters from generative process defined from equation can be inferred by EM algorithm. 
The graphical model induced by the generative process contains hidden states $Z$ and observations at
each node. This similar to HMM just instead of Markov chain, induced graph is a tree. Therefore EM algorithm 
is similar to Baum-Welch algorithm \cite{rabiner1986introduction}. 


The upward message $\alpha$ at node $t$ can be calculated starting from leafs and propagating upward. 
The observed variables influencing $Y_t$ are split into two subsets:
a) $e_Y_t^-$ composed of observed variables emitted by descendants of node $t$ (including t)  and
b) $e_Y_t^+$ composed of observed variable emitted by non-descendants of node $t$ (excluding t) \cite{starr2004introduction}.
$l$ and $r$ are respectively left and right child of node $t$.
$p$ and $s$ are respectively parent and sibling of node $t$.
$\oplus$ is outer product of two vector.
\begin{eqnarray}
	 \alpha_t(i) &=& Pr(e_Y_t^- |Z_t=i) \nonumber\\
	 &=& \sum_{i,j} \alpha_{l}(i) \alpha_{r}(j) \Gamma_{ij;k}  Pr_k(Y_t) \nonumber 
\end{eqnarray}
Where, $\Gamma_{ij;k}$ is transition matrix from $(Z_l = i , Z_r=j) $ to $Z_t=k$. $Pr_k(Y_t) = Pr(Y_t|Z_t =k)$
is the emission probability.
This can be written in following matrix form. It is important to keep these matrices information so that algorithm can be 
implemented in higher language program like R or matlab without sacrificing much on speed.
\begin{eqnarray}
	\boldsymbol{\alpha_{t}} &=& \boldsymbol{\delta P}(x_i)  \quad \forall t \in \texit{leaf} \nonumber\\ 
	\boldsymbol{\alpha_t} &=& \boldsymbol{(\alpha_l \oplus \alpha_r) \Gamma Pr}(Y_t)  \quad\forall t \not{\in} \texit{leaf}
	\label{eqn:alpha}
\end{eqnarray}

In the similar manner, we can derive the iterative formula for downward messages. 
\begin{eqnarray}
	 \beta_t(i) &=& Pr(e_Y_t^+ |Z_t=i) \nonumber\\
	 &=& \sum_{j,k} \alpha_{s}(j) \Gamma_{ij;k} \beta_{p}(k) Pr_i(Y_t) \nonumber\\ 
	 \texit{Equivalently },  \nonumber\\
	 \boldsymbol{\beta_t} &=&   \boldsymbol{\alpha_{s}} \Gamma_{ij;k}  \boldsymbol{\beta_{p} Pr(Y_t)} 
\end{eqnarray}

\begin{eqnarray*}
	\textit{E step} \quad
	\hat{v}_{jk;i}(t) &=&  Pr(C_{l} = j, C_{r} = k,  C_{t} = i | \boldsymbol{y^{T}}) \\
	  &=&  \alpha_l(j) \alpha_r(k) \gamma_{jk;i} p_k(x_t) \beta_t(k)/L_t\\
	  \textit{M step:} \quad \gamma_{jk;i} &=&\tilde{f}_{jk;i} \\
	  f_{jk;i} &=& \sum \hat{v}_{jk;i}(t)
\end{eqnarray*}

\section{Results}
\subsection{Simulation result}
\subsection{Comparison with Non-negative matrix factorization}
\subsection{GO annotation}
\subsection{Epigenetic enrichment}
\section{Implementation}

\bibliographystyle{apalike}	% (uses file "plain.bst")
\bibliography{diary}		% expects file "myrefs.bib"
\end{document}  

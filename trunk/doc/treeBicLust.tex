\documentclass{article}
%
%\usepackage{pdfsync}           % Used in Mac OSX
\usepackage[mathcal,mathbf]{euler}
\usepackage{theorem,amsmath,enumerate,fancyhdr,amssymb,amsfonts}
\usepackage{graphicx}           %Used in Mac OSX
%\usepackage[pdftex]{graphics}  %Might be needed in non Mac OSX systems
\usepackage{myDefs}
\usepackage{url}
%In order to be consistent use the following notation in your notes:
\graphicspath{{./fig/}}
\title{
Tissue tree gene biclustering and gene dynamics
}
\author{Avinash Das}

\date{\today}
\begin{document}

\pagestyle{fancy}
\lhead{{\bf Tissue tree biclustering}{ }
\\{\bf } } 
\rhead{{\bf Date: }\today}

\maketitle
\section{Abstract}
Problem of classification of genes is one of fundamental problem in field of genetics.
We propose a hierarchical  
Bayesian method to perform gene classification. We develop a message passing based 
EM  algorithm for inference. Final we demonstrate how the model can tease apart
gene dynamics on tissue tree. 
%\section{Problem}
%Given the gene expression across multiple tissues, find gene clusters and most probable tree defining relations between tissues.  
%Tissue tree implies the relation of the different tissues in an embryo, with embryo as root. Cluster of gene implies  
%group of genes homogeneously all active or inactive in tissues of subtree. For eg. one of intended outcome would be genes those
%are homogeneously active genes in tissue subtree with heart as the root, i.e. these genes will be active in left ventricle, right ventricle
%and all tissues related to heart.


%\section{Data}
%The data is collected from http://www.eurexpress.org/, composed of expression of 5600 genes across 811 tissues. The genes are manually annotated and have discreet value as strong, medium, weak or no expression. 
%For initial analysis we can treat the data as binary i.e. expressed or not expressed. Later it can be generalized to take four discreet values. 

%\section{Generative process}

%%\subsection{Defining latent variable Z}
%\par \textbf{Latent state variable $Z$}:
%Given a tree $\pi(V,E)$ and genes $G$, we define latent state variable $Z= \{z_{ju} \in\{0,1,2\} : \forall j \in G, u \in V\}$ as:
%%corresponding to each gene at each node:
%\begin{eqnarray}
	%z_{ju} &=& 0 \implies \textit{gene $j$ is homogenously inactive in subtree with $u$ as root.} \nonumber\\ 
	%z_{ju} &=& 1 \implies \textit{gene $j$ is homogenously active with high expression in subtree with  $u$ as root.} \nonumber \\
	%z_{ju} &=& 2 \implies \textit{gene $j$ have inhomogenously activity in subtree with $u$ as root.} \nonumber 
%\end{eqnarray}
%In addition, once a node $z_{ju}$ takes a value either 0 or 1 corresponding to states homogeneously active or in-active all its children 
%node in subtree stays in same homogeneous state. 

%Prior over tissue tree $\pi$ can be defined by Coalescent clustering. 

%$Z$ can be then sampled from multinomial distribution with a Dirichlet prior.
%\begin{eqnarray}
	%\pi &\sim& \text{coalescent}()\\
	%Z|\pi,\phi &\sim& \text{mult}(\phi) \nonumber \\
	%\phi &\sim&  \text{Dir}(\alpha,\beta,\gamma) \nonumber
	%\label{eqn:zdist1}
%\end{eqnarray}

%\par \textbf{Generation of expression level}: 
%Given the state variable $Z$, gene expression is generated from a Bernoulli with parameter only dependent on state variable $Z$. 
%\begin{eqnarray}
	%Y|Z = i &\sim& \text{Bin}(p_i) \nonumber \\
	%p_i &\sim&  \text{Beta}(\alpha_i,  \beta_i) \forall i \in \{1,2,3\}  
	%\label{eqn:ydist}
%\end{eqnarray}
%The beta priors can be chosen so to reflect our expected behaviour of expression level in different gene state. One example of prior that can be taken
%is shown in fig.% \ref{fig:betaP}.
%%\begin{figure}[ht]
	%%\begin{center}
		%%\includegraphics[scale=0.5]{beta.jpg}
	%%\end{center}
	%%\caption{Beta priors for homogeneous active, homogeneous inactive and inhomogeneous gene state. The priors are consistent
	%%with the interaction matrix data; e.g. $E(p_2)$ = average that any gene is expressed in interaction matrix.}
	%%\label{fig:betaP}
%%\end{figure}


%%The method can be extendedfor the discrete values of expression level by using multinomial distribution instead of binomial distribution in order to represent the different expression stages (weak, intermediate, high and no expression). 
%\section{Expected outcomes}
%\begin{itemize}
	%\item We will obtain a tissue tree. 
	%\item At each node of the tree we will get cluster of genes which are active or inactive in the subtree.
	%\item $\delta$ parameter in Kingman's coalescent will give the distance between the each tissue. For eg. we will get information if brain is closer to heart
		%than kidney. 
	%\item Probably clustering model will be resistant to error due manual annotation.
%\end{itemize}


%%% for presentation
%\begin{eqnarray}
	%\pi &\sim& \text{kingsman}() \nonumber\\
	%Z|\pi,\phi &\sim& \text{mult}(\phi) \nonumber \\
	%\phi &\sim&  \text{Dir}(\alpha,\beta,\gamma) \nonumber\\
	%Y|Z = i &\sim& \text{Bin}(p_i) \nonumber \\
	%p_i &\sim&  \text{Beta}(\alpha_i,  \beta_i) \nonumber
	%\label{eqn:pptzdist1}
%\end{eqnarray}
%\begin{eqnarray}
	%q(\rho_{l_i r_i}, \delta_i) \propto exp\left( -\binom{n-i +1}{2}\delta_i \right) \tilde{Z}_{i}(X|\theta_i) \nonumber \\
	%\tilde{Z}_{i}(X|\theta_i) = \int \int p(a) k_{- \infty t_{i}} (x,y) M_{i}(y) dy da \nonumber\\ 
	%M_i(y) \propto \prod_{b=l,r} \int k_{t_i t_{bi}}(y,y_b) M_i(y_b) dy_b \nonumber
%\end{eqnarray}

%\begin{equation}
	%Y|Z,\pi &\sim& \text{Beta}(\alpha_i, \beta_i) \nonumber
%\end{equation}

%\begin{eqnarray}
	%\boldsymbol{\alpha_{i}} &=& \boldsymbol{\delta P}(x_i)  \quad \forall i \in \textit{leaf} \\ 
	%\boldsymbol{\alpha_i} &=& \boldsymbol{\alpha_l \alpha_r \Gamma P}(x_i)  \quad\forall i \not{\in} \textit{leaf} 
%\end{eqnarray}

%\begin{eqnarray*}
	%\textit{E step} \quad
	%\hat{v}_{jk;i}(t) &=&  Pr(C_{l} = j, C_{r} = k,  C_{t} = i | \boldsymbol{y^{T}}) \\
	  %&=&  \alpha_l(j) \alpha_r(k) \gamma_{jk;i} p_k(x_t) \beta_t(k)/L_t\\
	  %\textit{M step:} \quad \gamma_{jk;i} &=&\tilde{f}_{jk;i} \\
	  %f_{jk;i} &=& \sum \hat{v}_{jk;i}(t)
%\end{eqnarray*}


\section{Introduction}
\subsection{Classification of genes}
One of the classical problem in the area of computational biology is to classify genes. 
There are more than 40 thousand known genes in human genome.
Function of many genes are 
poorly understood. 
%Further, function of many of these genes are not known or poorly understood.
They function in different manner in different context.  Many 
of the these genes control the expression of other genes and sometime regulate its own 
expression. This add a layer of complexity to analyze and define classification
of genes.


The task of gene classification is also important. Many of the genome wide studies like
genome-wide association studies or differential gene expresssion analysis find set
of genes. With better classification, we can analyze such experiment
Gene annotation is one example of several existing gene classification methods. It define 
tree based annotation of genes. It also give relationship of genes with pathways. 

Clustering is other popular method for gene classification. Clustering of genes tries
to find set of genes which are co-expressed. Co-expressed are sometime defined as two or more
genes
are expressed together across all samples. This is limiting definition for classification.
Genes are known to have different function in different context. Therefore, it is arguably
more useful to define clusters as 
set of genes which are co-expressed in subset of the samples.
Biclustering is approach to find significant submatrices in matrix. Biclustering of genes
therefore will fish out co-expressed genes in subset of samples.

The main limitation of clustering or biclustering approach is, they are local search algorithm
and hence may miss out important genes clusters. Another assumption that samples usually 
have flat relation. Many times samples corresponds to tissues or disease sample or samples 
from different individuals. All such cases trees is better representation of relationships.


Another very useful classification of gene classification is: a) housekeeping genes: genes
expressed in all tissues. b) tissue specific genes: genes which are expressed in few tissues.
The later category are gene which considered as gene signature of tissue. The main limitation
of these classification they are very simplistic. 

We propose a novel method of gene classification that combines these two approaches of 
gene classification. We try to use inherent tree relationship between sample to impute
better cluster. Further, we define homogeneous set of gene at each internal node of inferred
tree.


\subsection{In-situ hybridization data}

The data is collected from http://www.eurexpress.org/, composed of expression of 5600 genes across 811 tissues. The genes are manually annotated and have discreet value as strong, medium, weak or no expression. 
For initial analysis we can treat the data as binary i.e. expressed or not expressed. Later it can be generalized to take four discreet values. 


\section{Methods}

\subsection{Kingman's coalescent}
To infer tree from observed data, we used agglomerative clustering using Kingman's coalescent \cite{teh2008bayesian}.
It defines exponential distribution as prior over trees.  It calculates posterior of the trees given the observation.
The inference is performed by message passing algorithm, by passing message upward from leaf to trees. One of the advantage of
method is the distribution is exchangeable.

Let $\pi$ be the tree defining genaology of $n$ individuals. The tree $\pi$ can be defined by $n-1$ coalescent events. 
The $i$th coalescing event is defined by $\rho_{l_i}$ and $\rho_{r_i}$, the left and right subtree that are coalescing  
at waiting $\delta_i$ after $i-1$th coalescing event. The prior over tree was defined as:
\begin{equation}
	p(\pi) = \prod_{i}^{n-1}exp\left( -\binom{n-i +1}{2}\delta_i \right)
	\label{ref:prior}
\end{equation}


\cite{teh2008bayesian} proved that joint probability of observation and tree can be given by:
\begin{eqnarray}
	p(x, \pi) &\propto& \prod_{i}^{n-1}exp\left( -\binom{n-i +1}{2}\delta_i \right) \tilde{Z}_{i}(X|\theta_i) \\ 
	\textit{where, } \tilde{Z}_{i}(X|\theta_i) & =& \int \int p(a) k_{- \infty t_{i}} (x,y) M_{i}(y) dy da \\ 
	M_i(y) &\propto& \prod_{b=l,r} \int k_{t_i t_{bi}}(y,y_b) M_i(y_b) dy_b 
	\label{eqn:proposal}
\end{eqnarray}
$M_i(y)$ is message propagated upward to subtree $\theta_i$ from its both children. This can be calculated iteratively
by propagating messages from leaf to root node. $\tilde{Z}_{i}(X|\theta_i)$ can be viewed as local likelihood. Inference
is based on equation \ref{eqn:proposal}. At each step $i$, a duration is $\delta_i$ is sampled and then a pair $\rho_{l_i},\rho_{r_i}$
is chosen from the proposal distribution.


\subsection{Latent gene state variable $Z$}
Given we have reasonable accurate tissue tree representing the relationship between tissues,
the problem of gene biclustering reduces to inferring the 
state of genes at each internal node of the inferred tree. 
Given a tree $\pi(V,E)$ and genes $G$, we define latent state variable $Z= \{z_{ju} \in\{0,1,2\} : \forall j \in G, u \in V\}$ as:
%corresponding to each gene at each node:
\begin{eqnarray}
	z_{ju} &=& 0 \implies \textit{gene $j$ is homogenously inactive in subtree with $u$ as root.} \nonumber\\ 
	z_{ju} &=& 1 \implies \textit{gene $j$ is homogenously active with high expression in subtree with  $u$ as root.} \nonumber \\
	z_{ju} &=& 2 \implies \textit{gene $j$ have hetrogenous activity in subtree with $u$ as root.} \nonumber 
\end{eqnarray}
Once a node $z_{ju}$ takes a value either 0 or 1 corresponding to states  
homogeneously active or in-active all its children node in subtree stays in same homogeneous state.

\subsection{Generative process I}
The problem of biclustering of the genes 
reduces to simultaneously inferring 1) tree of tissue and 2) gene state variable at each internal node of the tree.
We define a generative process to generate gene expression data given the tree using $Z$ as the internal latent variable:
\begin{eqnarray}
	\pi &\sim& \text{kingsman}() \nonumber\\
	Z|\pi,\phi &\sim& \text{mult}(\phi) \nonumber \\
	\phi &\sim&  \text{Dir}(\alpha,\beta,\gamma) \nonumber\\
	Y|Z = i &\sim& \text{Bin}(p_i) \nonumber \\
	p_i &\sim&  \text{Beta}(shape1_i,  shape2_i)
	\label{eqn:gen1}
\end{eqnarray}
The basic idea of generative model is very similar to Kingman's agglomerative clustering. The main difference is instead of using 
observed data (gene expression) to infer tree, the generative process uses the state variable $Z$ at each internal nodes. 
Therefore, gene expression is thought to be generated at leafs given the tree and gene state variable.


The message passing algorithm in \cite{teh2008bayesian} uses transition kernel $ k_{t_i t_{bi}}(y,y_b)$
to define transition from one state to other. If 
$Z$ is used to infer tree, transitions cannot be independent for both children. That is transition of left children to
parent cannot be independent of state of right children. For instance:  once a node become homogeneous all it descendant 
should be homogeneous. This kind of complicated dependencies demand a three dimensional transition kernel. Such kernel does not
arrest a closed form solution of messages from equation \ref{eqn:proposal}.

\subsection{Generative process II}
To solve the problem described in previous section, we propose to split the tree inference with the inference of latent 
gene state $Z$. We infer tree directly from the observed gene expression from agglomerative clustering \cite{teh2008bayesian}.
At each of the internal state we also infer gene expression $Y$ at each internal nodes of tree by passing message downward and 
combing it with upward message. 

Given the tree $\pi$ and inferred gene expression at each internal node. We define a new generative process:

\begin{equation}
	Y|Z = i, \pi \sim \text{Beta}(shape1_i,  shape2_i)
	\label{eqn:gen2}
\end{equation}

The $Y$ at internal nodes are probability therefore it arrest a $Beta$ distribution. We can also assume given
the tree structure and expression at internal nodes, gene expression becomes independent. This implies
we can infer $Z$ independently for each genes. 

\subsection{Inference}
We used message passing algorithm to compute posterior probabilities of latent variable at internal nodes.
Parameters from generative process defined from equation can be inferred by EM algorithm. 
The graphical model induced by the generative process contains hidden states $Z$ and observations at
each node. This similar to HMM just instead of Markov chain, induced graph is a tree. Therefore EM algorithm 
is similar to Baum-Welch algorithm \cite{rabiner1986introduction}. 


The upward message $\alpha$ at node $t$ can be calculated starting from leafs and propagating upward. 
The observed variables influencing $Y_t$ are split into two subsets:
a) $e_{Y_t}^-$ composed of observed variables emitted by descendants of node $t$ (including t)  and
b) $e_{Y_t}^+$ composed of observed variable emitted by non-descendants of node $t$ (excluding t) \cite{starr2004introduction}.
$l$ and $r$ are respectively left and right child of node $t$.
$p$ and $s$ are respectively parent and sibling of node $t$.
$\oplus$ is outer product of two vector.
\begin{eqnarray}
	\alpha_t(i) &=& Pr(e_{Y_t}^- |Z_t=i) \nonumber\\
	 &=& \sum_{i,j} \alpha_{l}(i) \alpha_{r}(j) \Gamma_{ij;k}  Pr_k(Y_t) \nonumber 
\end{eqnarray}
Where, $\Gamma_{ij;k}$ is transition matrix from $(Z_l = i , Z_r=j) $ to $Z_t=k$. $Pr_k(Y_t) = Pr(Y_t|Z_t =k)$
is the emission probability.
This can be written in following matrix form. It is important to keep these matrices information so that algorithm can be 
implemented in higher language program like R or matlab without sacrificing much on speed.
\begin{eqnarray}
	\boldsymbol{\alpha_t} &=& \boldsymbol{\delta P}(x_i)  \quad \forall t \in \textit{leaf} \nonumber\\ 
	\boldsymbol{\alpha_t} &=& \boldsymbol{(\alpha_l \oplus \alpha_r) \Gamma Pr}(Y_t)  \quad\forall t \not{\in} \textit{leaf}
	\label{eqn:alpha}
\end{eqnarray}

In the similar manner, we can derive the iterative formula for downward messages. 
\begin{eqnarray}
	\beta_t(i) &=& Pr(e_{Y_t}^+ |Z_t=i) \nonumber\\
	 &=& \sum_{j,k} \alpha_{s}(j) \Gamma_{ij;k} \beta_{p}(k) Pr_i(Y_t) \nonumber\\ 
	 \textit{Equivalently },  \nonumber\\
	 \boldsymbol{\beta_t} &=&   \boldsymbol{\alpha_{s}} \Gamma_{ij;k}  \boldsymbol{\beta_{p} Pr(Y_t)} 
\end{eqnarray}
The complete likelihood of observed data is expressed in terms of $\alpha$ and $\beta$ as:
$ L_T= \alpha_t \beta_t' = Pr(\boldsymbol{Y^{(T)}} )$, for each $t$.   

The unknown parameters transition and emission probabilities can be inferred by EM algorithm:
\begin{eqnarray*}
	\textit{E step:} \quad
	\hat{u}_j(t) &=&  Pr(Z_t =  j/y^{(T)}) = \alpha_t(j) \beta_t(j) /L_T  \\
	\hat{v}_{jk;i}(t) &=&  Pr(Z_{l} = j, Z_{r} = k,  Z_{t} = i | \boldsymbol{y^{(T)}}) \\
\end{eqnarray*}
\begin{eqnarray*}
	  \textit{M step:} \quad
	  \gamma_{jk;i} &=&\tilde{f}_{jk;i} \\
	  f_{jk;i} &=& \sum \hat{v}_{jk;i}(t)
\end{eqnarray*}
Where, $\tilde{f}$ is normalized $f$. The shape parameter of emission probability can be found using the Newton's method that 
involves digamma and trigamma function. 
\section{Results}
\subsection{Convergence}
The EM algorithm converge rapidly. Typically it takes less than 50 iteration to converge (change in likelihood $<$ 1e-6). 
\begin{figure}[htbp]
	\begin{center}
		\includegraphics[scale=0.4]{../result/2013-05-08/convergence.jpg}
	\end{center}
	\caption{Rapid convergence of EM algorithm}
	\label{fig:convergence}
\end{figure}

We expect once as a node in the tissue tree is homogeneous active state all its descendant must be in active state. 
Therefore $\gamma_{11;1} \approx 1$. We expect similar behaviour for homogeneous inactive state. This seems to be the 
case for the transition matrix.
\begin{figure}[h]
	\begin{center}
		\includegraphics[scale=0.6]{../result/2013-05-08/transition.jpg}
	\end{center}
	\caption{Transistion matrix of 3 state HMM.}
	\label{fig:transistion}
\end{figure}


The emission probability of three states are consistent with definition of latent state variable
of gene $Z$.
\begin{figure}[h]
	\begin{center}
		\includegraphics[scale=0.6]{../result/2013-05-08/emission.jpg}
	\end{center}
	\caption{Emission probability from the homogeneously inactive(black), heterogeneous (red) and homogeneously active (green)}
	\label{fig:emission}
\end{figure}


%In particular, at the root of tree most of genes are expected to 
%be in hetrogenous state, therefore we expect beta distribution of heterogeneous should be
%agree with distribution of the original gene expression data. This result seems to be  hetrogenous state which is exp
\subsection{Simulation result}
We generated a simulated data set with the proposed generative process. Then, by keeping uniform
distribution as transistion matrix and random initialization parameter. We learned parameter 
by EM algorithm. The fig shows original transition matrix and inferred transition matrix.
\begin{figure}[h]
	\begin{center}
		\includegraphics[scale=0.3]{../result/2013-05-08/transition_orig.jpg}
		\includegraphics[scale=0.3]{../result/2013-05-08/transition_sim.jpg}
	\end{center}
	\caption{Transistion matrix of simulated data and estimated transition transition matrix}
	\label{fig:transistion2}
\end{figure}
\subsection{Gene dynamics}

The proposed method can be run on Tissue tree individually based on the independence assumption.
The state variable of gene gives its dynamics in the tissue tree. Fig. shows the dynamics
of one gene HMISC which knwn to e cardio vascular related. The agglomerative clustering nicely 
cluster together cluster corresponding to heart specific tissue. The swictching event is 
orange circle represent where transition probability convert to homogenous active stat from
hetrogenous state.

\begin{figure}[h]
	\begin{center}
		\includegraphics[scale=0.6]{../result/2013-05-08/final_result.jpg}
	\end{center}
	\caption{Emission probability from the homogeneously inactive(black), heterogeneous (red) and homogeneously active (green)}
	\label{fig:dynamics}
\end{figure}

%\subsection{Comparison with Non-negative matrix factorization}
%\subsection{GO annotation}
%\subsection{Epigenetic enrichment}
%\section{Implementation}

\bibliographystyle{apalike}	% (uses file "plain.bst")
\bibliography{diary}		% expects file "myrefs.bib"
\end{document}  

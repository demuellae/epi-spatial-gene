\documentclass{article}
%
%\usepackage{pdfsync}           % Used in Mac OSX
\usepackage[mathcal,mathbf]{euler}
\usepackage{theorem,amsmath,enumerate,fancyhdr,amssymb,amsfonts}
\usepackage{graphicx}           %Used in Mac OSX
%\usepackage[pdftex]{graphics}  %Might be needed in non Mac OSX systems
\usepackage{myDefs}
\usepackage{url}
%In order to be consistent use the following notation in your notes:
\graphicspath{{./fig/}}
\title{
Tissue tree specific gene biclustering
}
\author{Avinash Das, Fabian Muller, Peter Ebert}

\date{\today}
\begin{document}

\pagestyle{fancy}
\lhead{{\bf Tissue tree specific biclustering}{ }
\\{\bf } } 
\rhead{{\bf Date: }\today}

\maketitle
\section{Background}

\cite{Kaski2010} gave a probabilistic generative model framework to find bicluster in the gene expression 
matrix (Gene x condition). The biclustering method can be seen as a particular instance of hierarchical biclustering,
where biclusters
are arranged in tree hierarchy. The bicluster near the roots are coarse group of condition tied by subset genes
with homogeneous expression across the group. 

As other generative modelling, they started by presenting a generative process to 
observe a gene expression matrix. The generative process first samples the tree hierarchy 
of conditions (i.e. the condition are partitioned into tree structure), next 
it positions the genes into the sampled tree structure and finally it samples expression of genes
for each condition (which are observed variables). 

\par \textbf{Sampling of tree structure}:
In order to partition the condition, they used the Chinese restaurant process (CRP). The root of the tree
is initialized with all the conditions in the data and are partitioned using CRP. Each of these groups are 
recursively partitioned till maximum tree height is reached. 

\par \textbf{Length assignment to edges}:
Next, they used feature activation model to find active genes at each node, defined by latent variable $z_{j,u}$ for
gene $j$ and node $u$.
The length $l_{uv}$ of each edge $(u,v)$ is sampled with $ \sim Beta(\alpha,\beta)$. 
The $z_{j,u}$ 
is then sampled with probability of gene to be active equal to $l_{uv}$ (i.e. $P(z_{j,u} = 1) = l_{uv}$). 
Further, once a gene is active at a node, it will remain active in all its children.  


\par \textbf{Generation of gene expression}
Given the gene states at each node of the tree, the gene expression $Y$ is generated using:
\begin{eqnarray}
	Y_{ju}|z_{ju}=1  &\sim& N(\mu_{ju}, \sigma^2) \nonumber\\
	Y_{ju}|z_{ju}=0  &\sim& N(\mu_{0}=1, \sigma_0^2=1) 
	\label{eqn:Kexp}
\end{eqnarray}

The generative process is shown in the fig. ~\ref{fig:Kgen}.
 
Next, they define joint distribution over the random variables and perform inferencing using the Gibbs
sampling.

\par In this context, our biclustering problem differs from their problem in following sense:
\begin{itemize}
	\item We already have tissue trees.
	\item Our expression data are discrete values ( no expression, weak expression, medium expression and
		high expression).
	\item In our tissue tree internal nodes have observation.
	\item We want, in addition to find gene clusters, to categorize genes based on its expression patter 
		across tissue.
\end{itemize}
\section{Method}
Taking cues from the generative defined \cite{Kaski2010}, we can define a generative process in context of our problem. We will start with the tissue tree $T(V,E)$. We will determine state of genes at each tree node. Finally, given the state of gene we will sample it expression level, which is 
observed in our case. 
%We will define set of the genes $G$ as features at the nodes of the tissue tree. 
\par \textbf{Latent variable $Z$}:
We define latent state variable $Z= \{z_{ju} \in\{0,1,2\} : \forall j \in G, u \in V\}$ corresponding to each gene at each node with following definition:
\begin{eqnarray}
	z_{ju} &=& 0 \implies \textit{gene $j$ is homogenously inactive in subtree with $u$ as root.} \nonumber\\ 
	z_{ju} &=& 1 \implies \textit{gene $j$ is homogenously active with high expression in subtree with  $u$ as root.} \nonumber \\
	z_{ju} &=& 2 \implies \textit{gene $j$ have inhomogenously activity in subtree with $u$ as root.} \nonumber 
\end{eqnarray}
In addition, once a node $z_{ju}$ takes a value either 0 or 1 corresponding to states homogeneously active or in-active all its children 
node in subtree stays in same homogeneous state. $Z$ can be then sampled from multinomial distribution with a Dirichlet prior.
\begin{eqnarray}
	z &\sim& mult(\phi) \nonumber \\
	\phi &\sim&  Dir(\alpha,\beta,\gamma) \nonumber
	\label{eqn:zdist}
\end{eqnarray}



\par \textbf{Generation of expression level}: Lets assume we have binary data for expression level i.e. only 2 state 
expressed or not expressed. Given the state of the genes at the nodes of the tissue tree,  we can generate samples of expression
level using a binomial distribution. 
\begin{eqnarray}
	Y|Z = i &\sim& Bin(p_i) \nonumber \\
	p_i &\sim&  Beta(\alpha_i,  \beta_i) \forall i \in \{1,2,3\}  
	\label{eqn:ydist}
\end{eqnarray}
The beta priors can be chosen so to reflect our expected behaviour of expression level in different gene state. One example of prior that can be taken
is shown in fig. \ref{fig:betaP}.
\begin{figure}[ht]
	\begin{center}
		\includegraphics[scale=0.5]{beta.jpg}
	\end{center}
	\caption{Beta priors for homogeneous active, homogeneous inactive and inhomogeneous gene state. The priors are consistent
	with the interaction matrix data; e.g. $E(p_2)$ = average that any gene is expressed in interaction matrix.}
	\label{fig:betaP}
\end{figure}


We can extend the method for the discrete values of expression level by using multinomial distribution instead of binomial distribution. 
\subsection{Inference}
TBD.
\section{Expected outcomes}
\begin{itemize}
	\item Tissue specific gene active and inactive gene bicluster: We will get state of gene $Z$ for each tissue subtree, that can be used to 
		get information like gene clusters with tissue specific activity. We will also get information whether a 
		gene cluster is active or inactive homogeneously at that node.
	\item We can  quickly identify gene which are only expressed only at a node (or small subtree) and are not expressed in rest of the
		tree. 
	\item We can quantify distance between different node of tissue tree in term of expression level i.e we will get a distance between
		different tissues. 
	\item We will obtain a noise resistant model because we are modelling the output observation as distribution.   
\end{itemize}

\bibliographystyle{apalike}	% (uses file "plain.bst")
\bibliography{diary}		% expects file "myrefs.bib"
\end{document}  
